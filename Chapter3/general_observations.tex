\section{General Observations}
\label{sec:general_observations}

Before moving on to a large-scale engineering system a few general observations regarding the application of collocation-based and Kriging reduced order models to the uncertainty quantification of the  reactor systems described are in order. Primarily, the demonstration problems indicate that use of only 1D components in the anchored-\ac{ANOVA} expansion provides a very good approximation to the true system under investigation. From a computational point of view, 1D components are cheap to build and their quantity is equal to the number of random variables modeled. Higher order components generally require higher levels of interpolation and so their construction should be minimized if possible. As mentioned in \cite{AHSGC_HighDimensions}, the interaction effects between random variables for most realistic physical systems have a negligible effect on outputs of interest. 

Nevertheless, if multivariate components are needed algorithm \ref{code:rom_algorithm} is able to identify the combination of components most likely to affect the output, which offers significant computational savings. For example, if all components of two random variables were to be included in the anchored-\ac{ANOVA} expansion then 4950 two-dimensional sparse grid interpolants would need to be built. However, if only 5 of the 100 variables are deemed "important" by algorithm \ref{code:rom_algorithm} then only 10 two-dimensional sparse grid interpolants will be built. Not to mention, identification of variables that most affect the variability in some computer code output of interest offers insight into the physical system under consideration. It should be reemphasized that components of the anchored-\ac{ANOVA} used to build reduced order models do not represent the order of effects on the output. Even the 1D component functions can model nonlinear behavior as discussed in section \ref{subsec:anchored_anova}. Rather, the multivariate components describe the effect of input variables upon some output when acting together.    

For the demonstration problems investigated in this chapter, the Clenshaw-Curtis collocation abscissas performed significantly better than Gauss-Patterson. The Clenshaw-Curtis knots offered effectively the same convergence rates as Gauss-Patterson with far fewer function evaluations. Not to mention, Clenshaw-Curtis knots are much easier to generate.

The application of Morris' Algorithm for design variable screening was found to be a visually useful tool for dimension reduction. Whether a computational problem has a large number of dimensions or not the results of Morris' Algorithm can be viewed on a two-dimensional plot. Design variables having the greatest influence on an objective output's behavior become instantly recognizable. After identifying each example problem's key design variables, a reduced order model based on surrogate Kriging was constructed. The Kriging models' ability to reproduce statistics calculated used exact and collocation-based models using only twenty or so basis points demonstrated great promise for further applications.       

However, in all of the example problems analyzed in the current chapter the input uncertainties were calculated systematically beforehand and were therefore relatively small. In other words, engineering judgment never had to be applied in order to estimate the uncertainty of some model input parameter. Uncertainty estimates based on engineering judgment are reasonably expected to be significantly greater than those that can be calculated because more information is known about the latter. The introduction of large uncertainty estimates will be expected in the proceeding application, largely due to the fact that the sources of uncertainty are unknown. Such estimates are expected to negatively impact the smooth construction of surrogate models to represent the applications of interest.          


 

