\begin{frame}
\frametitle{Background}

\begin{itemize}
  \item Physical phenomena is studied by conducting experiments. 
  \item Any data collected represents instances of underlying stochastic processes.  
  \item We build predictive computer models in an attempt to reproduce such observed physical phenomena.
  \item To accurately capture stochastic element of experiments, computer models should be probabilistic.  
  \item In other words, inputs to computer models have uncertainties associated with them that are propagated to any outputs of interest.
  \item Running computer simulations should be like conducting physical experiments. Computer experiments.      
\end{itemize}

\end{frame}
%%%%%%%%%%%%%%%%%%%%%%%%%%%%%%%%%%%%%%%%%%%%%%%%%%%%%%%%%%%%%%%%%%%%%%%%%%%%%%%%%%%%%%%%
\begin{frame}
\frametitle{Why Surrogates?}

\begin{itemize}
  \item We run computer simulations to meet design objectives under certain constraints. 
  \item Involves numerical optimization, calibration, and performing what-if analyses.
  \item Also, we're interested in determining which of our design variables have the greatest effects on simulation outcomes. 
  \item Thousands of simulations required to make this possible but...  
  \item Computer simulations that model real phenomenon like nuclear reactors often take $\mathcal{O}(\mbox{hours})$ or $\mathcal{O}(\mbox{days})$ to complete.
  \item Building a surrogate model for your expensive computer simulations can make everything listed above possible.   
\end{itemize}

\end{frame}
%%%%%%%%%%%%%%%%%%%%%%%%%%%%%%%%%%%%%%%%%%%%%%%%%%%%%%%%%%%%%%%%%%%%%%%%%%%%%%%%%%%%%%%%
\begin{frame}
\frametitle{What is a Surrogate Model?}

\begin{itemize}
  \item A model for the outcomes of (likely) expensive computer simulations that can be rapidly evaluated while simultaneously preserving the predictive capabilities of the original simulations.
  \item Want to intelligently choose subspace $\lbrace x_1, x_2, ..., x_N\rbrace \subset \mathbf{X}$ to sample expensive computer code to get $\lbrace y_1, y_2, ..., y_N\rbrace \subset \mathbf{Y}$.
  \item Learn fast mapping that approximates $\mathbf{X}\rightarrow\mathbf{Y}$.   
\end{itemize}

\begin{figure}
\begin{tikzpicture}[node distance=1cm, auto]  
\tikzset{
    mynode/.style={rectangle,rounded corners,draw=black, top color=white, bottom color=blue!50,very thick, inner sep=1em, minimum size=3em, text centered},
    myarrow/.style={->, >=latex', shorten >=1pt, thick},
    mylabel/.style={text width=7em, text centered} 
}  
\node[mynode, text width=2cm, label=$\mathbf{X}$] (design_vars) {Design Variables};  
\node[mynode, right= of design_vars, text width=2cm, label=(Black Box)] (computer_code) {Expensive Computer Code};
\node[mynode, right= of computer_code, text width=2cm, label=$\mathbf{Y}$] (outcomes) {Outcomes};
\draw[myarrow] (design_vars) edge node {} (computer_code);
\draw[myarrow] (computer_code) edge node {} (outcomes);
\end{tikzpicture}  
\end{figure}

\end{frame}

%%%%%%%%%%%%%%%%%%%%%%%%%%%%%%%%%%%%%%%%%%%%%%%%%%%%%%%%%%%%%%%%%%%%%%%%%%%%%%%%%%%%%%%%
\subsection{Proposed Application}
%%%%%%%%%%%%%%%%%%%%%%%%%%%%%%%%%%%%%%%%%%%%%%%%%%%%%%%%%%%%%%%%%%%%%%%%%%%%%%%%%%%%%%%%
\begin{frame}
\frametitle{Proposed Application to Fuel Performance Modeling}

\begin{itemize}
  \item Fission Gas Release (FGR) refers to the phenomenon where Xenon and Krypton gases formed in UO$_2$ fuel rods are released into the rod free volume.
  \item Causes pressure build-up and thermal conductivity degradation in the rod filling gas, potentially jeopardizing the safety of the reactor.
  \item Fission gas atoms generated in the fuel grains diffuse towards the grain boundaries. 
  \item Majority of the gas diffuses into grain-face gas bubbles, giving rise to grain-face swelling.
  \item Bubble growth brings about bubble coalescence and interconnection, eventually leading to the formation of a tunnel network through which the fission gas is released.       
\end{itemize}

\end{frame}   
%%%%%%%%%%%%%%%%%%%%%%%%%%%%%%%%%%%%%%%%%%%%%%%%%%%%%%%%%%%%%%%%%%%%%%%%%%%%%%%%%%%%%%%%
\begin{frame}
\frametitle{Bison: Fuel Perfromance Modeling Code}  

\begin{columns}
 \begin{column}{0.6\textwidth}

\begin{itemize}
  \item A finite-element based, parallel, fully-coupled nuclear fuel performance code under development at Idaho National Laboratory.
  \item Models complex, multiphysics phenomena occurring over distances ranging from inter-atomic spacing to meters, time scales from $\mu s$ to years. 
  \item Simulations have required as many as 12,000 CPUs. Bison is as expensive as codes get.    
\end{itemize}

 \end{column}

 \begin{column}{0.4\textwidth}
  \centering
  \includegraphics[width=1.\textwidth, totalheight=0.45\textheight]{./bison_mesh.png} \\
  \includegraphics[width=1.\textwidth, totalheight=0.45\textheight]{./bison_is_stud.png}
 \end{column}
\end{columns}

\footnotetext[1]{\tiny Richard Williamson et. al. Overview of the BISON Multidimensional Fuel Performance Code. IAEA Technical Meeting: Modeling of Water-Cooled Fuel Including Design-Basis and Severe Accidents. Oct. 28, Chengdu, China.}

\end{frame}
%%%%%%%%%%%%%%%%%%%%%%%%%%%%%%%%%%%%%%%%%%%%%%%%%%%%%%%%%%%%%%%%%%%%%%%%%%%%%%%%%%%%%%%%
\begin{frame}
\frametitle{SIFGRS FGR Model}

\begin{itemize}
  \item Simple Integrated Fission Gas Release and Swelling (SIFGRS)
  \item Incorporates gas diffusion and precipitation in grains, growth and coalescence of gas bubbles at grain faces, thermal, athermal, steady-state, and transient gas release. 
  \item Through a direct description of the grain face gas bubble development, the fission gas swelling and release are calculated as coupled processes.
  \item Parameterized by, among others, linear heat rate, gas diffusion coefficient, surface tension of grain face bubbles, hydrostatic pressure, fuel grain radius, fuel porosity, and grain boundary sweeping. 
\end{itemize}

\end{frame}
%%%%%%%%%%%%%%%%%%%%%%%%%%%%%%%%%%%%%%%%%%%%%%%%%%%%%%%%%%%%%%%%%%%%%%%%%%%%%%%%%%%%%%%%
\begin{frame}
\frametitle{Ris\o~AN3 Experiment}

\begin{itemize}
  \item Validation case for fuel performance modeling in the Fumex-II database.
  \item Experiment consists of a base irradiation of four reactor cycles in the Biblis A pressurized water reactor.
  \item After the base irradiation period, a fuel rod is extracted and refabricated to a shorter length before undergoing a power ramp.
  \item Refabricated fuel rod is outfitted with various instrumentation such that fuel centerline temperature, FGR and rod internal pressure measurements can be obtained.    
\end{itemize}

\end{frame}
%%%%%%%%%%%%%%%%%%%%%%%%%%%%%%%%%%%%%%%%%%%%%%%%%%%%%%%%%%%%%%%%%%%%%%%%%%%%%%%%%%%%%%%%
\begin{frame}
\frametitle{Ris\o~AN3 Experiment Irradiation Profiles}

\begin{columns}
 \begin{column}{0.5\textwidth}
  \centering
  Base Irradiation History
  \includegraphics[width=1.\textwidth]{./base_irrad.png}
 \end{column}
 \begin{column}{0.5\textwidth}
  \centering
  Power Ramp Experiment \\ (This is what we're modeling)
  \includegraphics[width=1.\textwidth]{./power_ramp.png}
 \end{column}
\end{columns}

\end{frame}
%%%%%%%%%%%%%%%%%%%%%%%%%%%%%%%%%%%%%%%%%%%%%%%%%%%%%%%%%%%%%%%%%%%%%%%%%%%%%%%%%%%%%%%%
\begin{frame}
\frametitle{Modeling Ris\o~AN3 Experiment with Bison}

\begin{columns}
 \begin{column}{0.7\textwidth}

\begin{itemize}
  \item Fuel rod modeled as two fuel pellets smeared together.
  \item The first pellet has a hole down the center. Mesh consists of 29 axial nodes and 10 radial nodes.
  \item Second pellet has no hole down the center. Mesh consists of 166 axial nodes and 13 radial nodes.
  \item The clad mesh consists of 131 axial nodes and 3 radial nodes. 
  \item A 2-dimensional axi-symmetric quadratic element mesh used. 
\end{itemize}

 \end{column}
 \begin{column}{0.3\textwidth}
  \centering
  Bison Mesh
  \includegraphics[width=1.\textwidth]{./riso_an3_mesh.png}
 \end{column}
\end{columns}

\end{frame}
%%%%%%%%%%%%%%%%%%%%%%%%%%%%%%%%%%%%%%%%%%%%%%%%%%%%%%%%%%%%%%%%%%%%%%%%%%%%%%%%%%%%%%%%
%\begin{frame}
%\frametitle{Modeling Ris\o~AN3 Experiment with Bison}
%
%\begin{itemize}
%  \item SIFGRS parameters are quite generic and uncertain. 
%  \item Bison over-predicts FGR by a factor of 2 some 40 hours into the power ramp.
%\end{itemize}
%
%\begin{columns}
% \begin{column}{0.5\textwidth}
%  \centering
%  Fission Gas Release
%  \includegraphics[width=1.\textwidth]{./fgr_comparison.png}
% \end{column}
% \begin{column}{0.5\textwidth}
%  \centering
%  Fuel Centerline Temperature
%  \includegraphics[width=1.\textwidth]{./tc_temp_comparison.png}
% \end{column}
%\end{columns}
%
%\end{frame}
%%%%%%%%%%%%%%%%%%%%%%%%%%%%%%%%%%%%%%%%%%%%%%%%%%%%%%%%%%%%%%%%%%%%%%%%%%%%%%%%%%%%%%%%
\begin{frame}
\frametitle{Ris\o~AN3 Experiment}

\begin{itemize}
  \item Bison does a good job predicting fuel centerline temperature, very strongly correlated to power.
\end{itemize}

\begin{columns}
 \begin{column}{0.5\textwidth}
  \centering
  Power Ramp Experiment
  \includegraphics[width=1.\textwidth]{./power_ramp.png}
 \end{column}
 \begin{column}{0.5\textwidth}
  \centering
  Fuel Centerline Temperature
  \includegraphics[width=1.\textwidth]{./tc_temp_comparison.png}
 \end{column}
\end{columns}

\end{frame}
%%%%%%%%%%%%%%%%%%%%%%%%%%%%%%%%%%%%%%%%%%%%%%%%%%%%%%%%%%%%%%%%%%%%%%%%%%%%%%%%%%%%%%%%
\begin{frame}
\frametitle{Ris\o~AN3 Experiment}

\begin{itemize}
  \item Bison over-predicts FGR by a factor of 2 some 40 hours into the power ramp.
\end{itemize}

\begin{columns}
 \begin{column}{0.5\textwidth}
  \centering
  Power Ramp Experiment
  \includegraphics[width=1.\textwidth]{./power_ramp.png}
 \end{column}
 \begin{column}{0.5\textwidth}
  \centering
  Fission Gas Release
  \includegraphics[width=1.\textwidth]{./fgr_comparison.png}
 \end{column}
\end{columns}

\end{frame}
%%%%%%%%%%%%%%%%%%%%%%%%%%%%%%%%%%%%%%%%%%%%%%%%%%%%%%%%%%%%%%%%%%%%%%%%%%%%%%%%%%%%%%%%
\begin{frame}
\frametitle{Modeling Ris\o~AN3 Experiment with Bison}

\begin{itemize}
  \item Bison predictions of FGR and temperature fields stand to be improved by calibrating FGR parameters to experimental data.
  \item Calibration studies require $\mathcal{O}(10^3)$ function evaluations, which in this research is the Bison computer code.
  \item Each simulation of the Ris\o~AN3 experiment will take a few hours on multiple processors.
  \item It's necessary to construct a surrogate for the calibration study. 
\end{itemize}

\end{frame}
