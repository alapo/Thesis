
%%%%%%%%%%%%%%%%%%%%%%%%%%%%%%%%%%%%%%%%%%%%%%%%%%%%%%%%%%%%%%%%%%%%%%%%%%%%%%%%%%%%%%%%
\begin{frame}
\frametitle{Summary of Calibration Results}

\begin{itemize}
  \item There are significant discrepancies between predicted and experimental time series, especially in the power burst occurring at hour fifty of the power ramp.
  \item For the case when raw experimental data is used there is only a 2.6\% relative error in the EOE FGR prediction. The BOE prediction error is 64.8\%. 
  \item For the case of smoothed experimental date the prediction results are marginally improved with a BOE prediction error of 57.8\% and an EOE error of 0.5\%. 
  \item It's possible to enforce the conditions of matching the predicted BOE and EOE predictions to their respective experimental values in the COBYLA framework. 
  \item Enforcing only one of the conditions to a tolerance of $10^{-3}$ was achievable although the resulting solution grossly over predicted the fission gas release elsewhere in the time series.
\end{itemize}

\end{frame}
%%%%%%%%%%%%%%%%%%%%%%%%%%%%%%%%%%%%%%%%%%%%%%%%%%%%%%%%%%%%%%%%%%%%%%%%%%%%%%%%%%%%%%%%
\begin{frame}
\frametitle{Differences Between Prediction and Experiment}

\begin{itemize}
  \item There are apparent and irreconcilable differences between Bison's FGR predictions and the experimental data. 
  \item Some of these differences can be attributed to several fission gas release aspects not explicitly modeled in Bison. 
  \item Namely, burst fission gas release due to micro-cracking and the effect of measuring fission gas release using pressure transducers. 
  \item Uncertainty in SIFGRS parameters not modeled in this research.
\end{itemize}

\end{frame}
%%%%%%%%%%%%%%%%%%%%%%%%%%%%%%%%%%%%%%%%%%%%%%%%%%%%%%%%%%%%%%%%%%%%%%%%%%%%%%%%%%%%%%%%
\begin{frame}
\frametitle{Insights From Sensitivity Analysis}

\begin{itemize}
  \item The fuel grain radius and fuel temperature had the highest sensitivity indices and produced the largest non-linear interaction effects with the other parameters. 
  \item While sensitivity coefficients were calculated for the Ris\o~AN3 problem, it is not certain that the same parameter conclusions would generalize to other fission gas kinetics problems. 
  \item The same type of analysis as conducted here would likely have to be replicated for each unique problem. 
  \item The lack of generalization is likely due to the unique profundity of physics in play for each type of problem. 
\end{itemize}

\end{frame}
%%%%%%%%%%%%%%%%%%%%%%%%%%%%%%%%%%%%%%%%%%%%%%%%%%%%%%%%%%%%%%%%%%%%%%%%%%%%%%%%%%%%%%%%
\begin{frame}
\frametitle{Why Kriging?}

\begin{itemize}
  \item Non-linearity of fission gas release models coupled with large uncertainties implied the need for modeling higher-order interaction effects. 
  \item Modeling such higher-order effects with anchored-ANOVA and Smolyak sparse grids can get very expensive, with no clear limit of how many objective function simulations will be needed. 
  \item Transparency of Kriging appealing when considering each Bison fission gas release simulation would have to be performed in parallel.
  \item If simulations fail to converge or experiences an error, the correction process is straight forward. Contrarily, there are a lot of moving pieces in the anchored-ANOVA surrogate approach. 
  \item Clear extension to time series. 
\end{itemize}

\end{frame}
%%%%%%%%%%%%%%%%%%%%%%%%%%%%%%%%%%%%%%%%%%%%%%%%%%%%%%%%%%%%%%%%%%%%%%%%%%%%%%%%%%%%%%%%
\begin{frame}
\frametitle{Original Contributions of Research}

\begin{itemize}
  \item Construction of a surrogate model for the fuel performance code Bison and subsequent calibration of fission gas release parameters to experimental data from the FUMEX database. 
  \item Extension of Kriging to construct surrogates for entire time series through PCA. 
  \item Application of anchored-ANOVA decomposition and Smolyak sparse grids to construct surrogates for classic nuclear engineering problems.
\end{itemize}

\end{frame}
%%%%%%%%%%%%%%%%%%%%%%%%%%%%%%%%%%%%%%%%%%%%%%%%%%%%%%%%%%%%%%%%%%%%%%%%%%%%%%%%%%%%%%%%
\begin{frame}
\frametitle{Questions?}

\end{frame}














