% Reduced-Order Models

Often an engineering team seeks to perform optimization and calibration routines on a set of computer codes which may require several hours, if not days, to complete a single simulation. Given the limited computational budget available to such teams a surrogate model is typically sought for the computational codes. The use of a surrogate model for an expensive computer code effectively exchanges predictive accuracy for simulation execution time. Unfortunately, to construct a surrogate model the expensive computer code must still be executed a certain number of times, although still significantly less times than required to conduct an optimization study. At this point there are two directions that can be taken to construct the surrogate model. If a computational budget is limited to say, $N$ evaluations of the objective function then it is necessary to somehow optimize the set of input points at which the objective computer code is evaluated. The general strategy for constructing a surrogate is outlined in Fig. \ref{fig:surrogate_process}, with each component described in detail in the proceeding sections. Sampling plan optimization strategies are described in section \ref{sec:sampling_plans}, while surrogates based on the resulting sampling plan, namely Kriging, is described in section \ref{sec:kriging}. If there is more flexibility in the computational budget then collocation-based surrogates should be considered since they offer the benefits of built-in convergence metrics.        

Computer codes that model physical phenomena typically accept an input file whose purpose is to describe specific conditions in the universe being modeled by the code. The code is executed and the affects of the conditions on some dependent quantities are output. From this perspective computer codes can be viewed and treated in much the same way functions mathematicians deal with are treated. Like matrices, functions with certain properties can undergo various decompositions that offer insight into their structure. The purpose of section \ref{sec:func_decomp} is to describe a technique for decomposing functions into orthogonal components, with the ultimate intention of applying the technique to computer codes. It is hoped that the decomposition of the computer code in terms of its inputs reveals which inputs play the most active roles in the underlying physics. Keeping only the most active dimensions in the decomposition, a reduced order model is effectively built. However, the function decomposition technique described in section \ref{sec:func_decomp} describes only half the story. To create a reduced order model of a computer code that can be efficiently evaluated at any state point in the original parameter space an efficient, multidimensional, interpolation scheme is needed. Such a scheme is discussed in section \ref{sec:smolyak_sg}. The coupling between interpolation and function decomposition that creates a usable reduced order model is then described in section \ref{sec:decomp_and_interp}.    
\begin{figure}
\caption{\label{fig:surrogate_process}
General flow diagram for constructing a surrogate for an expensive computer code.}
\begin{tikzpicture}[node distance = 2.5cm, auto]
\tikzstyle{every node}=[font=\small]
\node[block, label=right:(Define computer experiments)] (samp_plan) {Sampling Plan};
\node[block, below of= samp_plan, label=right:(Quantitative evaluation of designs)] (obs) {Observations};
\node[block, below of= obs, label=right:(Strategic processing of design variables)](dim_red) {Dimension Reduction};
\node[block, below of= dim_red, label=right:(Create fast mapping)](surr) {Surrogate Construction};
\node[block, below of= surr, label=right:(Evaluate surrogate many times)](opt) {Optimization};
\path [line] (samp_plan.south) -- (obs.north);
\path [line] (obs.south) -- (dim_red.north);
\path [line] (dim_red.south) -- (surr.north);
\path [line] (surr.south) -- (opt.north);
\end{tikzpicture}  
\end{figure}

% Optimized Sampling Plans
\section{Optimized Sampling Plans} \label{sec:sampling_plans}

The following section is concerned with the problem of identifying an optimal set of points at which to build a surrogate model for an expensive computer code when only $N$ evaluations can be afforded. All surrogate models are built around a set of points at which the objective computer code is actually evaluated. Intuitively, the surrogate accuracy is expected to decrease as one moves further away from such points. Consequently, it is important to spread $N$ points as uniformly as possible across the design space. 

\subsection{Morris Algorithm}
\label{subsec:morris_algorithm}

Expensive computer codes generally have many input parameters because they attempt to model some phenomena on a very fine scale as accurately as possible. Of course, when writing such computer codes engineers are not aware which input parameters have the greatest impact on the outputs of interest or else only these parameters would be modeled. Contrarily, engineers typically only know that certain parameters are involved in the pertinent physics in some way but not to what extent. Due to the curse of dimensionality, the less design variables considered in the construction of a surrogate the cheaper the computational cost will be \cite{Forrester}. Consequently, before attempting to construct a surrogate for an expensive computer code it is worth identifying which design variables are most active. After all, if a design variable has a trivial effect on some output of interest then its presence in the surrogate should be minimized, for this is the very purpose of a surrogate model. One method for weeding out unimportant design variables is described by Morris \cite{Morris}, which is summarized in algorithm \ref{code:morris_algorithm}.   

The premise of Morris' algorithm is that if the derivative of some output parameter with respect to a design variable changes significantly throughout the design space then the variable is important. If the output parameter does not change with respect to the design variable then the variable can safely be ignored. To this end, the metric in Eq. \ref{eq:elementary_effect_x} is introduced by Morris to estimate the so-called elementary effect $d_i\left(\textbf{x}\right)$ of design variable $x_i$.
\begin{equation}
\label{eq:elementary_effect_x}
   d_i\left(\textbf{x}\right) = \frac{f\left(x_1,x_2,...,x_{i-1},x_i+\Delta,x_{i+1},....,x_k 										\right) - f\left(\textbf{x}\right)}
   								{\Delta}      
\end{equation}  
In Eq. \ref{eq:elementary_effect_x}, $\Delta$ is a step-length size for the perturbation. For convenience all variables $x_i$ are normalized to unit length and divided into $p$ segments such that $x_i\in\lbrace 0,1/(p-1),2/(p-1),...,1\rbrace$. Choosing a set of $\textbf{x}$ carefully, it is possible to calculate an elementary effect for each of $k$ design variables using only $k+1$ function evaluations. Indeed, as described in \cite{Morris}, a $[k+1]\times [k]$ random orientation matrix $B^*$ can be constructed using the equation,
\begin{equation}
\label{eq:random_orientation}
   \textbf{B}^* = \left(\textbf{1}_{k+1,1}\textbf{x}^* + \frac{\Delta}{2}\left[
                   \left(2\textbf{B} - \textbf{1}_{k+1,k}\right)\textbf{D}^* + 
                    \textbf{1}_{k+1,k}\right]\right)\textbf{P}^*.
\end{equation}            
In Eq. \ref{eq:random_orientation}, $\textbf{1}$ is a matrix of ones with size specified by its subscript and $\textbf{B}$ is a $[k+1]\times [k]$ matrix of zeros and ones with the characteristic that for each column there exists a pair of rows differing in only their $i^{th}$ entry for $i\in\lbrace 1,2,...,k\rbrace$. Also, $\textbf{D}^*$ is a $[k]\times [k]$ diagonal matrix with ones of differing parity uniformly spread, $\textbf{P}^*$ is a $[k]\times [k]$ random permutation matrix, and $\textbf{x}^*$ is a random point chosen in the $p$-level design space. When the rows of $\textbf{B}^*$ are evaluated by the objective function and substituted into Eq. \ref{eq:elementary_effect_x} an elementary effect is calculated for each design variable. 

The more elementary effects that can be calculated for each design variable, the better the estimate as to the effect of each design variable on the objective function. Consequently, $r$ random orientation matrices are typically created to obtain a total of $r$ elementary effects for each design variable. Taking the mean and standard deviation of each variable's $r$ effects can yield insight into the most important variables. Plotting the mean and standard deviation of each variable's effects on a scatter plot, variables that have a negligible effect on the objective function will cluster around the origin. Large fluctuations in standard deviation are indicative of nonlinear and interactive effects \cite{Morris}. 

\begin{algorithm}
\caption{\label{code:morris_algorithm} 
Uses Morris' Algorithm to determine which of a function's design variables induce the most significant effects and interactions.} 
\begin{algorithmic}[1]
\State Initialize zeros matrix $d_{stats}$ of size $[r]\times[k]$
\For {$i=1:r$}
   \State $X\rightarrow$ Create random orientation matrix 
   \Comment{Eq. \ref{eq:random_orientation}} 
   \State $f_{base} = f(X[0,:])$
   \For {$j=1:k$}
      \State $f_{new} = f(X[k,:])$
      \State $l\rightarrow$ Find index of effect being perturbed
      \State $d_{effect}\rightarrow$ Calculate elementary effect \Comment{Eq. \ref{eq:elementary_effect_x}} 
      \State $d_{stats}[i,l] = d_{effect}$
      \State $f_{base} := f_{new}$
   \EndFor
\EndFor
\State Return mean and standard deviation of $d_{stats}$ 
\end{algorithmic}
\end{algorithm}

\subsection{Latin Hypercube Sampling}
\label{subsec:lhs}

One of the major problems with random sampling occurs when a relatively limited sample size is utilized. In this case, subsets of the sample space with high consequence but low probability are likely to be missed \cite{Helton}. In addition, the proximity of sampled values caused by random sampling is inefficient, which often causes slow convergence. In order to resolve such issues \ac{LHS} was conjured. The basis of \ac{LHS} rests upon dividing the normalized space of each design variable into $n$ equally sized bins if $n$ samples are required. As a result, when the $n$ samples are taken it is guaranteed that the entire spectrum of each design variable's space has been visited. Algorithmically an $n$ sample Latin hypercube in $k$ dimensions can be easily calculated, as described in algorithm \ref{code:lhs}.

\begin{algorithm}
\caption{\label{code:lhs} 
Creates a random Latin hypercube consisting of $n$ samples in $k$ dimensions.} 
\begin{algorithmic}[1]
\State Initialize zeros matrix $X$ of size $[n]\times[k]$
\For {$i=1:k$}
   \State $p\rightarrow$ Create random permutation of the set $\lbrace 1,2,...,n\rbrace$ 
   \State $X[:,i] := p[:]$
\EndFor
\State Map each entry of $X$ into hypercube
\State Return $X$ 
\end{algorithmic}
\end{algorithm}

Each row of the output from algorithm \ref{code:lhs} is a sample point normalized to a hypercube. As mentioned previously, when sampling a design space it is important to take sample points uniformly. While \ac{LHS} increases the likelihood of obtaining such a uniform sampling space at random it is actually possible to obtain an optimized sampling space based on the maximin metric \cite{Forrester}.

The maximin metric describe by Morris and Mitchell \cite{Morris_Mitchell} makes use of two notions in an attempt to quantify the uniformity, or 'space-fillingness', of a set of sampling points. In order to describe the notions for each sampling plan it is useful to gather $\lbrace d_1, d_2, ..., d_m\rbrace$ and $\lbrace J_1, J_2, ..., J_m\rbrace$, the unique distances between all points in the plan sorted in ascending order and the number of occurrences of each distance, respectively. In words, the Morris and Mitchell criteria states that an optimized sampling plan will minimize all $J_i$ while maximizing the corresponding $d_i$. More formally, Morris and Mitchell define the maximin sampling plan as one which maximizes $d_1$, and among plans for which this is true, minimizes $J_1$, among plans for which this is true, maximizes $d_2$, and so on \cite{Forrester}. The previous definition can be restated into pseudo-equivalent minimization problem by introducing the parameter $\Phi_q(\textbf{X})$,
\begin{equation}
\label{eq:Phi_q}
   \Phi_q(\textbf{X}) = \left(\sum_{j=1}^m J_j d_j^{-q} \right)^{1/q}
\end{equation}
where $\textbf{X}$ is a sampling plan and $q$ is a control parameter inherent in the minimization problem. 

The minimization of Eq. \ref{eq:Phi_q} and the Morris and Mitchell definition of the maximin sampling plan are generally used in unison to obtain a locally optimal sampling plan since finding the globally optimal plan is computationally infeasible. In this approach a random sampling plan $\textbf{X}_0$ is initially generated using algorithm \ref{code:lhs}. Using a range of $q$ values, usually from one to one-hundred, an optimal latin hypercube plan is found for each value based on the initial sampling plan $\textbf{X}_0$. To obtain an optimized plan for each $q$ algorithms such as simulated annealing \cite{Kirkpatrick} and evolutionary operation \cite{Box} can be used. These algorithms work to minimize Eq. \ref{eq:Phi_q} by comparing an initial sampling plan to perturbed versions, which are obtained by switching pairs of column entries in the output of algorithm \ref{code:lhs}. Once an optimized \ac{LHS} plan is found for each $q$ value, the resulting set of plans are contested directly against each other by explicit application of Morris and Mitchell's maximin definition. The sampling plan satisfying the maximin criteria is the locally optimal sampling plan to be used for proceeding surrogate model construction. Algorithm \ref{code:best_lhs} summarizes the search for a locally optimal \ac{LHS} plan.  

\begin{algorithm}
\caption{\label{code:best_lhs} 
Obtains a locally optimal \ac{LHS} plan using the Morris-Mitchell minimax criteria.} 
\begin{algorithmic}[1]
\State Initialize sampling plan $\textbf{X}_0$ \Comment{Apply algorithm \ref{code:lhs}}
\State $q = \left[1, 2, 5, 10, 20, 50, 100\right]$
\State $\textbf{X}^{cands.} \rightarrow$ Initialize empty array for optimal $\textbf{X}$ candidates
\For {$q_i$ in $q$}
   \State $\textbf{X}_{opt}(q_i)\rightarrow$ Find optimal $\textbf{X}$ for $q_i$ using simulated annealing \Comment{Minimize Eq. \ref{eq:Phi_q}}
   \State Add $\textbf{X}_{opt}(q_i)$ to $\textbf{X}^{cands.}$ 
\EndFor
\State Apply Morris-Mitchell criterion between all $(\textbf{X}^{cands.}_i, \textbf{X}^{cands.}_j)$ to find optimal plan
\end{algorithmic}
\end{algorithm}
  


 


% Kriging
\section{Kriging} \label{sec:kriging}

Write about Kriging!
% Function Decompositions
\section{Function Decompositions} \label{sec:func_decomp}

In order to reduce the dimensionality of some function there must exist a metric to determine the importance of each dimension with respect to the others. It's important to have a framework that isolates the effects of each dimension on the output of the function. The framework chosen to perform dimension reduction is formally known as \ac{HDMR}. In statistics, the \ac{ANOVA} decomposition is a special case of \ac{HDMR} where the Lesbegue measure is used to perform all integrations.

\subsection{Dimension-wise Decompositions} \label{subsec:dim_wise_decomps}

The dimension-wise \ac{HDMR} is algorithmically similar to Gram-Schmidt for matrix orthogonalization in that orthogonal components are systemtically removed to create a linearly independent basis. As in Gram-Schmidt, the essential operator in dimension-wise \ac{HDMR} is the projection operator. Before introducing the projection operator of interest in this thesis, define the $d$-dimensional product measure to be,
\begin{equation} \label{eq:d_dimension_product_measure}
    d\mu\left(\textbf{x}\right) = \prod_{j=1}^{d}
                              d\mu_{j}\left(x_j\right) 
\end{equation}
where $\mu_j$ are probability measures defined over some $\Omega$. Two functions $f,g:\Omega^d \rightarrow \mathbb{R}$ are considered orthogonal with respect to the product measure defined in \ref{eq:d_dimension_product_measure} if the inner product,
\begin{equation} \label{eq:inner_product}
   \left(f,g\right)=\int_{\Omega^d} f(\textbf{x})g(\textbf{x})d\mu(\textbf{x})
\end{equation} 
is equal to zero. To introduce the projection operator $P_\textbf{u}$, the notation used in \cite{Holtz} is adopted. The operator $P_\textbf{u}$ projects from a $d$-dimensional space to a $\vert\textbf{u}\vert$-dimensional space for some set $\textbf{u}\subseteq \mathcal{D}$, where $\mathcal{D}=\lbrace 1,...,d\rbrace$ consists of the set of all coordinate indices in $\textbf{x}$. The projection on to $\textbf{u}$ is given as,
\begin{equation} \label{eq:projection_operator}
    P_\textbf{u}f\left(\textbf{x}_{\textbf{u}}\right) =
     \int_{\Omega^{d-\vert\textbf{u}\vert}} 
      f\left(\textbf{x}\right)d\mu_{\mathcal{D}\setminus\textbf{u}}
       \left(\textbf{x}\right)
\end{equation}
where $\textbf{x}_\textbf{u}$ has length $\vert\textbf{u}\vert$ and consists of the $\textbf{x}$ coordinates specified in $\textbf{u}$. Also, the notation $\mathcal{D}\setminus\textbf{u}$ signifies all the coordinates in $\mathcal{D}$ not contained in $\textbf{u}$. From \ref{eq:projection_operator} it is clear that the projection operator works to integrate out all coordinate indices not contained in $\textbf{u}$ from $f$. For some coordinate indice sets $\textbf{u}$ and $\textbf{v}$, where $\textbf{u}\neq\textbf{v}$, the following orthogonality relation holds,
\begin{equation} \label{eq:orthogonality_relation}
    \left(f_\textbf{u},f_\textbf{v}\right) = 0.
\end{equation}
The notation $f_\textbf{u}$ is used to denote the function of only the coordinate indices contained in $\textbf{u}$. From \ref{eq:orthogonality_relation}, it follows that a function can be written in terms of its $2^d$ orthogonal components,
\begin{equation} \label{eq:hdmr_decomp}
    f\left(\textbf{x}\right) = 
     \sum_{\textbf{u}\subseteq\mathcal{D}}
      f_\textbf{u}\left(\textbf{x}_\textbf{u}\right)
\end{equation}
where the component functions $f_\textbf{u}$ are defined recursively as \cite{Holtz},
\begin{equation} \label{eq:recurvsive_component_def}
    f_\textbf{u}\left(\textbf{x}_\textbf{u}\right) = 
     P_\textbf{u}f\left(\textbf{x}_\textbf{u}\right) -
      \sum_{\textbf{v}\subset\textbf{u}}
       f_\textbf{v}\left(\textbf{x}_\textbf{v}\right).  
\end{equation}
The recursive definition in \ref{eq:recurvsive_component_def} can be written explicitly as,
\begin{equation} \label{eq:explicit_component_def}
    f_\textbf{u}\left(\textbf{x}_\textbf{u}\right) =
     \sum_{\textbf{v}\subseteq\textbf{u}}
      \left(-1\right)^{\vline\textbf{u}\vline-\vline\textbf{v}\vline}
       P_\textbf{v}f\left(\textbf{x}_\textbf{v}\right)    
\end{equation}

For most functions arising in engineering applications, especially if the function is a computer code, the decomposition in \ref{eq:hdmr_decomp} is not possible to obtain because each component function $f_\textbf{u}$ will require a high-dimensional integral to be performed. Of course, this statement assumes a Lebesgue measure in the definition of $d\mu$ in \ref{eq:d_dimension_product_measure}. Alternatively, if a Dirac measure is used then the computationally burdensom integral in \ref{eq:projection_operator} is reduced to a single function evaluation. In this case, the decomposition in \ref{eq:hdmr_decomp} is referred to as an anchored-\ac{ANOVA} decomposition, or CUT-\ac{HDMR} \cite{AHSGC_HighDimensions}.

\subsection{Anchored-ANOVA Decomposition} \label{subsec:anchored_anova}

Using the Dirac measure $\delta(\textbf{x}-\textbf{a})d\textbf{x}$ to evaluate the projection operator at a fixed point $\textbf{a} \in \left[0,1\right]^d$ in the hypercube, equation \ref{eq:projection_operator} becomes,
\begin{equation} \label{eq:projection_operator_dirac}
    P_\textbf{u}f\left(\textbf{x}_\textbf{u}\right) = 
     f\left(\textbf{x}\right)\vert_{\textbf{x}=
      \textbf{a}\setminus\textbf{x}_\textbf{u}}.
\end{equation}
The notation $\textbf{a}\setminus\textbf{x}_\textbf{u}$ is the anchor point $\textbf{a}$ except at the coordinate indices specified in $\textbf{u}$. At the coordinate indices $\textbf{u}$, the anchor point takes upon the corresponding values in $\textbf{x}$. Using the Dirac measure to evaluate the projections comprising \ref{eq:hdmr_decomp}, the objective function is expressed as a linear combination of its values along lines, faces, hyperplanes,..., etc \cite{Holtz}. As mentioned previously, using the Dirac measure to perform the projection operations in \ac{HDMR} results in enormous computational savings since high-dimensional integrals are replaced with single function evaluations.

Given the structure of anchored-\ac{ANOVA}, it is not surprising to learn that there exists a close connection to multivariate Taylor series\cite{HDMR}. This connection provides insight into some of the properties of the anchored-\ac{ANOVA} decomposition. The multivariate Taylor series of $f(\textbf{x})$ about a point $\textbf{\=x}$ can be written as,
\begin{equation} \label{eq:multivariate_Taylor_series}
    f\left(\textbf{x}\right) = 
     f\left(\bar{\textbf{x}}\right) +
      \sum_{i=1}^{d} 
       \frac{\partial f(\textbf{x})}{\partial x_i}\left(x_i - \bar{x}_i\right)
        + \frac{1}{2!}\sum_{i,j=1}^{d}
         \frac{\partial^2 f(\textbf{x})}{\partial x_i\partial x_j}
          \left(x_i - \bar{x}_i\right) \left(x_j - \bar{x}_j\right) + ...
\end{equation}       
As an example, consider what happens if \ref{eq:multivariate_Taylor_series} is evaluated at $\textbf{x} = \textbf{a}\setminus x_i$,
\begin{equation} \label{eq:Taylor_series_ex1}
    f\left(\textbf{x}\right)\vert_{\textbf{x} = \textbf{a}\setminus x_i} = 
     f\left(\bar{\textbf{x}}\right) +
      \frac{\partial f(\textbf{x})}{\partial x_i}\left(x_i - \bar{x}_i\right)
       + \frac{1}{2!} \frac{\partial^2 f(\textbf{x})}{\partial x_i^2}
        \left(x_i - \bar{x}_i\right)^2 + ...
\end{equation}
Since $f_i\left(x_i\right) = f\left(\textbf{x}\right)\vert_{\textbf{x} = \textbf{a}\setminus x_i} - f\left(\bar{\textbf{x}}\right)$, 
\begin{equation} \label{eq:Taylor_series_ex2}
    f_i\left(x_i\right) =
     \frac{\partial f(\textbf{x})}{\partial x_i}\left(x_i - \bar{x}_i\right)
       + \frac{1}{2!} \frac{\partial^2 f(\textbf{x})}{\partial x_i^2}
        \left(x_i - \bar{x}_i\right)^2 + ...      
\end{equation}  
Expression \ref{eq:Taylor_series_ex2} shows that the first-order component functions in anchored-\ac{ANOVA} consist of entire Taylor series expansions. Similarly, second-order component functions will consist of their respective entire Taylor series expansions and so on. Consequently, a truncated anchored-\ac{ANOVA} expansion will always provide a better approximation to a function than a truncated Taylor expansion \cite{HDMR}. 

\subsubsection{Effective Dimensions} \label{subsec:effective_dims}

The ultimate purpose of introducing an expansion such as anchored-\ac{ANOVA} is to truncate it and use the truncated portion as an approximation to the objective function. Of course, evaluation of the truncated anchored-\ac{ANOVA} expansion is expected to be much more computationally efficient than the objective function. When an anchored-\ac{ANOVA} decomposition is truncated, the loss incured becomes the components \ref{eq:recurvsive_component_def} that are not being represented. Of course, in practical construction the components not represented are calculated to contribute relatively trivially. Two notions exist for classifying the dimension of a truncated anchored-\ac{ANOVA} decomposition. Both notions depend on $\hat{\sigma}(f)$, which is the sum of the absolute values of the integrals of all anchored-\ac{ANOVA} terms \cite{Holtz}
\begin{equation} \label{eq:sum_all_integrated_components}
    \hat{\sigma}\left(f\right) = 
     \sum_{\begin{subarray}
     \textbf{u} \subseteq \mathcal{D} \\
     \textbf{u} \neq \emptyset
     \end{subarray}}
      \vert If_\textbf{u} \vert \approx
       \sum_{\begin{subarray}
       \textbf{u} \subseteq \mathcal{D} \\
       \textbf{u} \neq \emptyset
       \end{subarray}}
        \vert q_\textbf{u} \vert .
\end{equation}
The notation $I\cdot$ represents an exact integral but, in practice the integral will be evaluated using some multivariate quadrature scheme and so the exact integral's approximation is denoted by $q_\textbf{u} \approx If_\textbf{u}$. For some user-defined threshold $\alpha \in \left[0,1\right]$ the truncation and superposition dimensions of a truncated anchored-\ac{ANOVA} expansion can be defined. The truncation dimension attempts to quantify the importance of a certain number of dimensions $d_t$. Mathematically, the truncation dimension is the smallest integer $d_t$ such that,
\begin{equation} \label{eq:trunc_dimension}
    \sum_{
     \begin{subarray}{c}
     \textbf{u} \subseteq \lbrace1,...,d_t\rbrace \\
     \textbf{u} \neq \emptyset
     \end{subarray}
    } 
     \vert q_\textbf{u} \vert \geq \alpha \hat{\sigma}(f). \nonumber
\end{equation}
Contrastingly, the superposition dimension attempts to quantify the order of important dimensions $d_s$. Mathematically, the superposition dimension is the smallest dimension $d_s$ such that,
\begin{equation} \label{eq:sup_dimension}
    \sum_{
     \begin{subarray}{c}
     \vert\textbf{u}\vert \leq d_s \\
     \textbf{u} \neq \emptyset
     \end{subarray}
    } 
     \vert q_\textbf{u} \vert \geq \alpha \hat{\sigma}(f). \nonumber
\end{equation}
       
Both definitions for the effective definition of a truncated anchored-\ac{ANOVA} expansion can be directly related to the exact integral of the objective function $If$. Specifically, for the truncation dimension the following relation holds \cite{Holtz},
\begin{equation} \label{eq:trunc_dim_error}
    \vert If -  
     \sum_{\textbf{u} \subseteq \lbrace 1,...,d_t\rbrace}
      If_\textbf{u} 
    \vert \leq \left(1-\alpha\right) \hat{\sigma}\left(f\right).
\end{equation}
Similarly, for the superposition dimension the following inequality holds,
\begin{equation} \label{eq:superpos_dim_error}
    \vert If -  
     \sum_{\vert \textbf{u} \vert \leq d_s}
      If_\textbf{u} 
    \vert \leq \left(1-\alpha\right) \hat{\sigma}\left(f\right).
\end{equation}
Ineqalities \ref{eq:trunc_dim_error} and \ref{eq:superpos_dim_error} suggest that if all the anchored-\ac{ANOVA} terms are used then the exact integral of the objective function can be reproduced. However, in general the set of effective dimensions as determined by anchored-\ac{ANOVA} will not be equal to the set determined by a classic \ac{ANOVA} decomposition. The choice of the anchor point $\textbf{a}$ has a direct influence on the accuracy and truncation dimension of the anchored-\ac{ANOVA} expansion \cite{Hesthaven_AnchorPoint}. In \cite{Hesthaven_AnchorPoint} the authors argue that choosing the anchor point to be the centroid of the parameter space is an excellent choice for most applications. As such, in this thesis the anchor point is always chosen to be the centroid of the working parameter space $\Omega^d$.    

% Smolyak Sparse Grids
\section{Smolyak Sparse Grids} \label{sec:smolyak_sg}

In order to create a reduced-order model for some objective function the anchored-\ac{ANOVA} decomposition plays a crucial role but more is needed \cite{Hesthaven_ANOVA}. Recall that the primary purpose for constructing a reduced-order model is to replace the presumably computationally intensive objective function with something that is trivial to evaluate. Consequently, in evaluating the anchored-\ac{ANOVA} decomposition at some point $\textbf{x}$ the projections in  \ref{eq:projection_operator_dirac} must be trivial to evaluate as well. As it stands, evaluating the anchored-\ac{ANOVA} decomposition for some objective function is significantly more expensive than simply evaluating the function itself. To resolve this issue, a Smolyak sparse grid interpolant is created for each projection. While creating each such interpolant incurs some initial overhead, the payoff is the desired reduced order model. 

\subsection{Motivation} \label{subsec:motivation}

To describe multivariate function interpolation based on Smolyak sparse grids it makes sense to speak in the context of quadrature since a quadrature rule consists of interpolating a function using polynomials and then integrating the polynomials exactly. For the moment consider some smooth 1D function $f(x)$. The function $f(x)$ can be approximated arbitrarily well through the summation,
\begin{equation} \label{eq:1D_interp}
   f(x) \approx \sum_{i=1}^{P} f(x_i)C_i(x)
\end{equation}
where $C_i(x)$ are cardinal functions of degree $P$ with the property that $C_i(x_j)=\delta_{ij}$, $\delta_{ij}$ being the Kronecker $\delta$-function \cite{Boyd}. By the Weierstrass approximation theorem, smooth functions can be uniformly approximated as closely as desired by polynomial functions \cite{TrefethenApprox}. At the collocation points, or abscissas, in \ref{eq:1D_interp} the function $f(x)$ is interpolated exactly at $x_i$. The function $f(x)$ is comprised of various constant, linear, quadratic, cubic,..., etc terms and so exact integration of $f(x)$ amounts to integrating its monomial constituents.  

Suppose that instead of interpolating a 1D function, a multivariate function of $d$ dimensions is to be interpolated. The naive approach to multivariate interpolation is to take a Cartesian product of 1D rules, such as in \ref{eq:1D_interp}, $d$ times. Consequently, the product grid will contain $P^d$ points, each of which requires a unique function evaluation. Such exponential growth is coined the "curse of dimensionality" \cite{LeMaitreKnio}. As a rule of thumb, exact integration of a monomial constituent comes at the cost of a single function evaluation \cite{Burkardt_SCALA2013}. Considering the space of $d$-dimensional, $P$-degree polynomials has some,
\begin{equation} \label{eq:polynomial_space}
    \binom{P+d}{d} \approx \frac{d^P}{P!} 
\end{equation} 
dimensions, for high dimensional problems the full Cartesian product approach integrates a superfluous number of monomials. Russian mathematician Sergei Smolyak was one of the first to realize the potential computational savings in his paper \cite{SmolyakOriginal}.

\subsection{Algorithm Mechanics} \label{subsec:algorithm_mechanics}

A Smolyak sparse grid is the set of collocation points used to build an interpolant for some multivariate objective function while the Smolyak algorithm is the whole procedure of building the interpolant. To begin, the Smolyak algorithm will be stated and pertinent notation will be introduced. Since indice tracking comprises the brunt of understanding the Smolyak algorithm, it is crucial to choose a clear notation. Consequently, the notation used here closely follows that of \cite{NovakRitter}.

Slightly generalizing \ref{eq:1D_interp}, for the case of some smooth 1D function $f$, let $U^i$ be the interpolant of $f$ comprised of $m_i$ collocation points.
\begin{equation} \label{eq:1D_interp2}
    U^i = \sum_{j=1}^{m_i} 
     f\left(x_j^i\right) a_j^i
\end{equation}  
In \ref{eq:1D_interp2}, $i \in \mathbb{N}$, and $a_j^i \in C(\left[-1,1\right])$ are basis functions imposing the demand that $U^i$ exactly be able to reproduce $f$ at the collocation points $x_j^i$. The notation $x_j^i \in \left[-1,1\right]$ refers to the $j^{th}$ collocation point of $m_i$ total points. Restricting the domain of the collocation points to $\left[-1,1\right]$ does not impose any limitations on being able to interpolate $f$ arbitrarily well since $\left[-1,1\right]$ can always be mapped to the parameter space of $f$.

To generalize from 1D interpolation to multivariate interpolation 1D interpolation formulas, such as the one in \ref{eq:1D_interp2}, are combined using tensor products.
\begin{equation} \label{eq:muitivariate_interp_fulltp}
    \left(U^{i_1} \otimes\cdots\otimes U^{i_d}\right)\left(f\right) = 
     \sum_{j_1=1}^{m_{i_1}} \cdots
      \sum_{j_d=1}^{m_{i_d}} f\left(
       x_{j_1}^{i_1},\cdots,x_{j_d}^{i_d}\right)\cdot
        \left(a_{j_1}^{i_1}\otimes\cdots\otimes a_{j_d}^{i_d}\right)
\end{equation}   
Tensor products are a mathematical convenience used to represent all combinations of some entity, in this case $U^i$. The scheme in \ref{eq:muitivariate_interp_fulltp} suffers from the, "curse of dimensionality" since a total of,
\begin{equation}
    \prod_{k=1}^d m_{i_k}
\end{equation}     
function evaluations are needed to form the interpolant. The Smolyak algorithm is based on \ref{eq:muitivariate_interp_fulltp}, the only difference being not all the tensor products are used. In explicit form, the Smolyak formula for approximating the left-hand side of \ref{eq:muitivariate_interp_fulltp} is given as \cite{NovakRitter},
\begin{equation} \label{eq:Smolyak_formula1}
    A\left(q,d\right) = 
     \sum_{q-d+1\leq \vert \textbf{i}\vert\leq q}
      \left(-1\right)^{q-\vert\textbf{i}\vert}
       \binom{d-1}{q-\vert\textbf{i}\vert}
        \left(U^{i_1} \otimes\cdots\otimes U^{i_d}\right).
\end{equation}
Each entry $i_k$ in the vector $\textbf{i} \in \mathbb{N}^d$ contains the indice corresponding to the level of interpolation in dimension $k$. The more collocation points being utilized, the higher the level of interpolation since the interpolant becomes increasingly accurate. The magnitude of $\textbf{i}$ is $\vert\textbf{i}\vert = \vert i_1 +\cdots+ i_d\vert$. Since each $i_d \geq 1$, the variable $q \geq d$. The variable $q$ essentially keeps track of the level of interpolation of the Smolyak algorithm. As $q$ is increased, more tensor product combinations are allowed. From \ref{eq:Smolyak_formula1} it is clear that the Smolyak algorithm is able to reduce the total number of tensor product components by limiting the entries of $\textbf{i}$. Note, when performing analysis using the Smolyak algorithm the "interpolation level" typically refers to $q-d$ in order to ground the analysis at an interpolation level of zero.

The Smolyak formula in \ref{eq:Smolyak_formula1} can be rewritten in several ways, all of which use the idea of the incremental interpolant $\Delta^i$ defined as,
\begin{eqnarray} \label{eq:incremental_interpolant}
    U^0 &=& 0 \nonumber \\
    \Delta^i &=& U^i - U^{i-1}
\end{eqnarray}
The incremental interpolant operator is simply the difference between interpolants at two successive levels. Using the notion of the incremental interpolant, the Smolyak formula can be rewritten as,
\begin{equation} \label{eq:Smolyak_formula2}
    A\left(q,d\right) =
     \sum_{\vert\textbf{i}\vert\leq q}
      \left(\Delta^{i_1}\otimes\cdots\otimes\Delta^{i_d}\right)
\end{equation}   
At first sight, \ref{eq:Smolyak_formula1} and \ref{eq:Smolyak_formula2} seem inefficient since neither exposes the recursiveness inherent in the Smolyak formula. In other words, when moving from index $q$ to $q+1$ the work done to get to level $q$ is not lost. Rewriting the Smolyak formula in a recursive fashion is advantageous for implementation on a computer.
\begin{eqnarray}
\label{eq:Smolyak_formula3a}
    A(q,d) &=& A(q-1,d) + \Delta A(q,d) \\
\label{eq:Smolyak_formula3b}
    \Delta A(q,d) &=& \sum_{\vert\textbf{i}\vert = q}
     \left(\Delta^{i_1}\otimes\cdots\otimes\Delta^{i_d}\right)
\end{eqnarray}
While the Smolyak algorithm representation in \ref{eq:Smolyak_formula3a} has the advantage of being represented recursively, it does not provide any type of indicator for when the Smolyak sparse grid should be refined. Collocation points should be added to a Smolyak sparse grid until the resulting interpolant is able to reproduce the objective function to some user-defined threshold. The authors in \cite{AHSGC} are able to rewrite \ref{eq:Smolyak_formula3b} in terms of what's referred to as a hierarchical surplus, 
\begin{equation} \label{eq:hierarchical_surplus}
    \Delta A(q,d) = \sum_{\vert\textbf{i}\vert=q}
     \left(f(x_{j_1}^{i_1},...,x_{j_d}^{i_d}) - 
      A(q-1,d)(x_{j_1}^{i_1},...,x_{j_d}^{i_d})\right)\cdot
       \left(a_{j_1}^{i_1}\otimes\cdots\otimes a_{j_d}^{i_d}\right)
\end{equation}
which appears as the first term in the summation as the difference between the function value at the point $(x_{j_1}^{i_1},...,x_{j_d}^{i_d})$ and the Smolyak $q-1$ level interpolant value at the same point. Level $q$ of the Smolyak algorithm generally contains all the points comprising level $q-1$ plus some new collocation points. Consequently, the level $q$ Smolyak interpolant is expected to exactly evaluate any collocation points born in previous levels. The summation in \ref{eq:hierarchical_surplus} is taken over all the new points in level $q$ that have not appeared in level $q-1$ since the hierarchical surplus for these will be identically equal to zero. The hierarchical surpluses provide an indicator for how well the Smolyak algorithm is interpolating some objective function. If the hierarchical surpluses are decreasing with each successive level then the Smolyak algorithm is converging.

Following the notation in \cite{NovakRitter}, let $X^i=\lbrace x_1^i,...,x_{m_i}^i\rbrace$ be the collocation points comprising $U^i$. From \ref{eq:Smolyak_formula1}, the total number of collocation points in a Smolyak sparse grid can be written as,
\begin{equation} \label{eq:numer_points_in_smolyak}
    H(q,d) = \bigcup_{q-d+1\leq\vert\textbf{i}\vert\leq q}
     \left(X^{i_1}\times\cdots\times X^{i_d}\right).
\end{equation}     

\subsection{Basis and Collocation Points} \label{subsec:basis_and_points}

The exactness of the Smolyak algorithm is decided mainly by the choice of collocation points $x_{j_k}^{i_k}$ used to build $H(q,d)$. The basis functions $a_{j_k}^{i_k}$ work to weave the collocation points together. Gaussian quadrature is a favorite of many since with only $n+1$ collocation points, all polynomials of degree $2n+1$ or less can be integrated exactly \cite{NumAnyHenrici}. However, collocation points derived from Gaussian quadrature schemes are not nested in that $X^i \not\subset X^{i+1}$. Nestedness in the choice of collocation points is an essential feature for reducing the computational expense of applying the Smolyak algorithm. If nested collocation points are chosen for each $X^{i_k}$ then the Smolyak sparse grid will also be nested such that $H(q-1,d)\subset H(q,d)$ \cite{NovakRitter}. Consequently, when improving the Smolyak interpolant from level $q-1$ to level $q$ one will only have to evaluate the objective function at the points that are unique to $X^i$, which are given as $X_{\Delta}^{i} = X^i \setminus X^{i-1}$ \cite{AHSGC}.  The set of new points in level $q$ of a Smolyak sparse grid are given as,
\begin{equation} \label{eq:unique_points_ssg}
    \Delta H(q,d) = \bigcup_{\vert\textbf{i}\vert = q}
     X_{\Delta}^{i_1} \times\cdots\times X_{\Delta}^{i_d}.
\end{equation}   

A viable alternative to Gaussian quadrature collocation points for the Smolyak algorithm is to use Clenshaw-Curtis collocation points, which consist of the extrema of Chebyshev polynomials. While $n+1$ Clenshaw-Curtis abscissas can only exactly integrate polynomials of degree $n$, they have the advantage of being nested. Accuracy is sacrificed for nestedness in the Smolyak algorithm, at least in theory. In practice it has been shown that for most functions Clenshaw-Curtis quadrature performs almost on par to Gaussian quadrature \cite{TrefethenQuadrature}. In other words, the double accuracy of Gaussian quadrature is rarely realized. For some level $i$ the Clenshaw-Curtis collocation points are given by,
\begin{equation} \label{eq:cc_points}
    x_{j}^{i} = \left\{
     \begin{array}{cr}
       \cos\frac{\pi(j-1)}{m_i-1}   & j=1,...,m_i \text{ if } i>1 \\
       0   &  j=1 \text{ if } i=1
     \end{array}
   \right.
\end{equation}
In order for the level $i$ Clenshaw-Curtis abscissas to contain the level $i-1$    abscissas, a total of $2^{i-1}$ new points must be added. Consequently, the total number of abscissas appearing in the level $i$ Clenshaw-Curtis scheme is given as,
\begin{equation} \label{eq:cc_numpoints}
    m_i = \left\{
     \begin{array}{cr}
      2^{i-1}+1   & i>1 \\
      1   & i=1
     \end{array}
    \right.
\end{equation}

Another alternative to the Gaussian and Clenshaw-Curtis abscissas is Gauss-Patterson. The Gauss-Patterson set of collocation points are nested and provide  a polynomial exactness of $(3n-1)/2$ with $n$ points, which is right in between the exactness of Clenshaw-Curtis and Gaussian sets. Obtaining the Gauss-Patterson abscissas involves a rather convoluted, iterative process and so the reader is referred to \cite{GaussPatterson} to review the methodology and obtain tables of the actual points. The growth rule for Gauss-Patterson goes as $2^i-1$, which is some factor of two greater than the growth rule for Clenshaw-Curtis. In \cite{HesthavenGaussPatt}, the authors conclude the Gauss-Patterson collocation points are competitive with Clenshaw-Curtis when comparing the cost and accuracy of computing quadratures using the same number of function evaluations.       

To weave together the collocation points forming a Smolyak sparse grid, some type of basis function $a_{j}^i$ is needed, as defined in \ref{eq:1D_interp2}. Although the basis functions will be applied to multi-dimensional interpolation, the Smolyak algorithm conveniently scales 1D basis functions to multiple dimensions through the use of tensor products. One basis commonly used  in adaptive sparse grids is the linear hat basis function \cite{Agarwal}. For the scheme in \ref{eq:cc_numpoints} the linear hat functions are given as, 
\begin{eqnarray} \label{eq:linear_hat_basis}
    a_{1}^{1} &=& 1 \text{ for } i=1, \\
    a_{j}^{i} &=& \left\{
     \begin{array}{cc}
      1-(m_i-1)\vert x-x_j^i\vert   & \text{ if }
      \vert x-x_j^i\vert < 1/(m_i-1) \\
       0   & \text{ else } \nonumber
     \end{array}
    \right.
\end{eqnarray}
for $i>1$ and $j=1,...,m_i$. While the linear hat functions have the advantage of local support they are limited to relatively slow convergence due to their lack of curvature. Offering faster error decay are the global Lagrange characteristic polynomials, 
\begin{equation} \label{eq:lagrange_basis}
    a_j^i = \left\{
     \begin{array}{cc}
      1   & \text{ if } i=1 \\
      \displaystyle \prod_{\begin{subarray}{l}
              k=1 \\
              k \neq j
             \end{subarray}}^{m_i}
       \frac{x-x_k^i}{x_j^i-x_k^i}   & j=1,...,m_i \text{ for } i>1    
     \end{array}
    \right.
\end{equation}  
However, the Lagrange characteristic polynomials are plagued by the fact that each evaluation of \ref{eq:1D_interp2} requires $\mathcal{O}(m_i^2)$ operations and  often the computation is numerically unstable \cite{BaryCentIntrp}. To remedy these concerns, the barycentric form of Lagrange characteristic polynomials is used to form a basis. The barycentric Lagrange basis is given as,
\begin{equation} \label{eq:barycentric_basis}
    a_j^i = \left\{
     \begin{array}{cc}
      1   & \text{ if } i=1 \\
      \displaystyle \frac{\frac{w_j^i}{x-x_j^i} }
       {\sum_{j=0}^{m_i} 
        \frac{w_j^i}{x-x_j^i}}   & j=1,...,m_i \text{ for } i>1
     \end{array}
    \right.
\end{equation} 
where $w_j^i$ are barycentric weights defined by,
\begin{equation} \label{eq:barycentric_weights}
    w_j^i = \frac{1}
     {\displaystyle \prod_{k \neq j}\left(x_j^i - x_k^i\right)}
      \hspace{1 cm} j=1,...,m_i.
\end{equation}
For special collocation sets, such as Clenshaw-Curtis in \ref{eq:cc_points}, explicit forms exist for the barycentric weights. Generally, forming the weights is an $\mathcal{O}(m_i^2)$ operation and then evaluation of an interpolant based on the barycentric Lagrange basis is only a $\mathcal{O}(m_i)$ operation \cite{BaryCentIntrp}. With an explicit form in hand, evaluation of the barycentric Lagrange basis is significantly cheaper than the Lagrange basis. For the Clenshaw-Curtis collocation points in \ref{eq:cc_points}, the barycentric weights are given by \cite{ChebyType2},
\begin{equation} \label{eq:barycentric_weights4_cc}
    w_j^i = (-1)^{j+1}\delta_j^i \hspace{1 cm}
     \delta_j^i = \left\{
                   \begin{array}{cc}
                    .5   & j=1 \text{ or } j=m_i \\
                    1   & \text{ else }  
				   \end{array}
				  \right. .                   
\end{equation}      
From \ref{eq:barycentric_basis}, an apparent problem exists if the barycentric basis is to be evaluated at a collocation point. As \cite{BaryCentIntrp} explains, the problem can be circumvented by simply perturbing the value of $x$ by an $\epsilon$ on the order of machine precision. In this case, the numerator and denominator in \ref{eq:barycentric_basis} will effectively cancel each other such that $a_j^i=1$. The barycentric Lagrange basis is therefore numerically stable.  

\subsection{Exactness and Error Bounds} \label{subsec:exactness_error}

The exactness of the Smolyak algorithm is determined by the space of polynomials the algorithm is exact on. Since the 1D interpolation rules, on which the Smolyak algorithm is based on, can exactly interpolate certain polynomials it is not presumptuous to expect the Smolyak algorithm to exactly interpolate certain polynomial spaces. Using the collocation set in \ref{eq:cc_numpoints} and \ref{eq:cc_points} the Smolyak interpolant $A(q,d)$ is exact on \cite{NovakRitter},
\begin{equation} \label{eq:polynomial_exactness}
    \sum_{\vert\textbf{i}\vert = q}
     \mathbb{P}(m_{i_1}-1,1)\otimes\cdots\otimes
      \mathbb{P}(m_{i_d}-1,1)
\end{equation} 
where $\mathbb{P}(k,d)$ is the space of all polynomials in $d$ dimensions of total degree no greater than $k$. From \ref{eq:polynomial_exactness} it follows that the Smolyak interpolant for $q=d+P$ is exact for all polynomials of degree $P$. In other words, the effects of any monomials containing $x^l$ for $l\leq P$ will be captured by the Smolyak algorithm. Recall from \ref{eq:polynomial_space} that the degrees of freedom of $\mathbb{P}(P,d)$ goes as $d^P/P!$. Any method aiming to reproduce $\mathbb{P}$ requires at least this many function evaluations. Since the number of collocation points in a Smolyak grid for $A(d+P,d)$ goes as $2^{P}d^{P}/P!$ the dependence on dimension is said to be optimal \cite{NovakRitter}. However, the asymptotic growth of points also indicates that the Smolyak algorithm requires still excessive function evaluations to achieve polynomial exactness.     

Since the Smolyak algorithm is constructed using one-dimensional interpolation formulas, which all have error bounds, it is also possible to derive error bounds for a Smolyak interpolant $A(q=d+P,d)$. While the reader is instructed to consult \cite{NovakRitter} for a detailed derivation of the error bounds, they will nevertheless be stated here. Consider some $d$-variable function $f$ with continuous derivatives of order $P$ in each variable. The error in using a Smolyak interpolant to approximate $f$ can be given as,
\begin{equation} \label{eq:smolyak_error}
    \Vert f-A(d+P,d)(f)\Vert_{\infty} \leq 
     c_{d,P} M^{-P} (\log M)^{(P+2)(d-1)+1}
\end{equation}
where $M$ is the total number of knots used by $A(d+P,d)$ and $c_{d,P}$ is a constant depending on both $d$ and $P$. From \ref{eq:smolyak_error}, the error in the Smolyak interpolant heavily depends on the smoothness of the function being interpolated and on the total number of collocation points used to form the interpolant.       

\subsection{Computer Implementation} \label{subsec:smolyak_implementation}

To implement Smolyak's algorithm on a computer equations \ref{eq:Smolyak_formula3a} and \ref{eq:hierarchical_surplus} should be utilized since together they provide a recursive definition. Much of the implementation efforts are concerned with indice book keeping. Although the pseudocode for Smolyak's algorithm used in this thesis is provided here, the reader is directed to \cite{Klimke_Wohlmuth} for more elaborate details. Efficiency of the algorithm can be increased by pre-calculating the desired abscissas as in \ref{eq:cc_points}, the number of abscissas at a given level as in \ref{eq:cc_numpoints}, and corresponding barycentric weights as in \ref{eq:barycentric_weights4_cc}. A data structure for quick retrieval of the desired values is also necessary. Abscissa information is constantly being reused in Smolyak's algorithm and so it is inefficient to have to recalculate values each time.   
\begin{algorithm}
\caption{\label{code:smolyak_algorithm} Smolyak's algorithm for creating an interpolant for a function $f$ of $d$ dimensions. The algorithm will exit if the maximum Smolyak level is reached or if one of the convergence criteria is met.}  
\begin{algorithmic}[1]
\State {Create data structure that stores indice coordinates and hierarchical surplus.}
\For{$q=d,\text{maximum level}$}
   \For{$(i_1,...,i_d)$ in enumerations of $i_1+...+i_d=q$}
   \Comment{See Alg. \ref{code:enumeration1}}
      \For{$(j_1,...,j_d)$ in enumerations of $(i_1,...,i_d)$}
      \Comment{See Alg. \ref{code:enumeration2}}   
         \State {Turn each $(j_1,...,j_d)$ into a knot.} 
         \State {Categorize as either processed or unprocessed knot.}    
      \EndFor
      \State Evaluate $f(\text{unprocessed knots})$.
      \State Calculate hierarchical surplus at unprocessed knots. 
      \Comment{Eq. \ref{eq:hierarchical_surplus}}
      \State Archive newly processed knots.
   \EndFor
   \State Check for convergence.
\EndFor 
\end{algorithmic}
\end{algorithm}

The pseudocode for Smolyak's algorithm in algorithm \ref{code:smolyak_algorithm} will now be discussed in some detail. To initialize the algorithm a data structure must be created to store all information for each index in the sparse grid. For level $q$ of the Smolyak algorithm the summation in \ref{eq:Smolyak_formula3a} is over all sets $(i_1,i_2,...,i_d)$ such that $i_1 + i_2 + ... + i_d = q$. Each such set corresponds to a knot $(x^{i_1}_{j_{i_1}}, ..., x^{i_d}_{j_{i_d}})$ in the random space defined by the hypercube $\left[-1,1\right]^d$. The knot, function value at the knot, and the corresponding hierarchical surplus should all be stored in the data structure. 

The first loop in the pseudocode tells the code to keep increasing the interpolation level in the Smolyak algorithm until some maximum level is reached, which is specified by the user. The purpose of this loop is to make sure the algorithm ends eventually. Of course, other convergence criteria are in place in hope that the algorithm terminates long before the maximum level is reached. The second loop goes through all the enumerations of $(i_1,i_2,...,i_d)$ such that $i_1 + i_2 + ... + i_d = q$. The algorithm for producing such enumerations is provided in algorithm \ref{code:enumeration1}. The third loop takes each enumeration and again enumerates over each index to obtain each component in the tensor product appearing in the Smolyak formulation. An algorithm to execute this enumeration is provided in algorithm \ref{code:enumeration2}.

In the main body of the pseudocode each output from algorithm \ref{code:enumeration2} is first converted to a knot $(x^{i_1}_{j_{i_1}}, ..., x^{i_d}_{j_{i_d}})$. Each potential knot must then be sorted into one of two categories. The motivation for the two categories arises from the fact that the same knot may be expressed in several ways. Since each knot corresponds to a function value and hierarchical surplus, significant computational savings can be incurred by not reevaluating the objective function at the same knots. Consequently, each potential knot is binned into either a category of knots that have already been evaluated at $f$ or a category of unevaluated knots.

Once all components in the tensor product for a given $(i_1,i_2,...,i_d)$ have been converted and sorted, the unevaluated knots are processed. In this step of the algorithm the previously unevaluated knots should be evaluated at $f$ in parallel if possible since each evaluation is completely independent. The resulting functions values should then be used to compute the hierarchical surplus in \ref{eq:hierarchical_surplus} for each knot. Once the function value and hierarchical surplus is available for each new knot the results should be archived in the indice data structure.     

Finally, once the second loop is complete, the level of the Smolyak interpolant has been effectively increased and it's time to check whether additional levels are required based on user-defined convergence criteria. Perhaps the best indicator of a Smolyak interpolant's convergence is the maximum hierarchical surplus calculated for all newly processed knots at the current interpolation level. The hierarchical surplus is a measure of how well the interpolant is able to match the objective function and therefore, if the hierarchical surpluses being calculated are decreasing with each level the interpolant is converging. An additional convergence criteria includes comparison of the relative change in computed mean and variance between two successive interpolation levels. For this thesis, the Smolyak algorithm is terminated only after the maximum hierarchical surplus is below a certain threshold, the relative change in interpolant mean is below a threshold, and the relative change in variance does not exceed a threshold. 
% Combining Decomposition and Interpolation
\section{Combining Decomposition and Smolyak's Algorithm} \label{sec:decomp_and_interp}

As hinted at in the beginning of chapter \ref{chap:rom}, the Smolyak algorithm combines with the anchored-\ac{ANOVA} decomposition to create a reduced order model for any well behaving computer code. To see how the Smolyak algorithm fits into the functional decomposition described in this chapter, begin by substituting \ref{eq:explicit_component_def} into \ref{eq:hdmr_decomp} to get,
\begin{equation} \label{eq:decomp_smol1}
   f(\textbf{x}) = 
    \sum_{\textbf{u}\subseteq\mathcal{D}}
     \sum_{\textbf{v}\subseteq\textbf{u}} 
      (-1)^{\vert\textbf{u}\vert - \vert\textbf{v}\vert}
       P_{\textbf{v}} f(\textbf{x}_{\textbf{v}}).
\end{equation} 
Now, insert the Dirac projection operator from \ref{eq:projection_operator_dirac} into \ref{eq:decomp_smol1} to arrive at,  
\begin{equation} \label{eq:decomp_smol2}
   f(\textbf{x}) = 
    \sum_{\textbf{u}\subseteq\mathcal{D}}
     \sum_{\textbf{v}\subseteq\textbf{u}} 
      (-1)^{\vert\textbf{u}\vert - \vert\textbf{v}\vert}
       f(\textbf{x})\vert_{  
        \textbf{x}=\textbf{a}\setminus\textbf{x}_{\textbf{v}} }.   
\end{equation}
To create a reduced order model the set $\mathcal{D}$ is ultimately shrunk to only contain a subset of all the variables of $f$ but this is discussed later. The important aspect of \ref{eq:decomp_smol2} to realize is that in order to evaluate $ f(\textbf{x}) \vert_{\textbf{x} = \textbf{a} \setminus \textbf{x}_{\textbf{v}}}$ evaluation of the expensive computer code $f(\textbf{x})$ is required. Consequently, the desireable property of reduced order models, that of rapid evaluation, is not achieved in \ref{eq:decomp_smol2}. The remedy is to approximate each $ f(\textbf{x}) \vert_{\textbf{x} = \textbf{a} \setminus \textbf{x}_{\textbf{v}}}$ using Smolyak interpolants. Substituting \ref{eq:Smolyak_formula2} into \ref{eq:decomp_smol2}, the formulation for creating a reduced order model of $f$ is complete. 
\begin{equation} \label{eq:decomp_smol3}
   f(\textbf{x}) = 
    \sum_{\textbf{u}\subseteq\mathcal{D}}
     \sum_{\textbf{v}\subseteq\textbf{u}} 
      (-1)^{\vert\textbf{u}\vert - \vert\textbf{v}\vert}
       \sum_{\vert\textbf{i}\vert\leq q}\left(
        \Delta^{i_1}\otimes\cdots\otimes\Delta^{i_{\vert\textbf{v}\vert}}
         \right)  
\end{equation}
While there is initial overhead to create an interpolant for each component in \ref{eq:decomp_smol2} the result is quick evaluation of the reduced order model. Details regarding the implementation and application of \ref{eq:decomp_smol3} will be discussed in the proceeding sections.               
 
\subsection{Implementation Details} \label{subsec:implement_details}
%Discuss combinatorics routines needed(2, list in appendix), algorithm, sampling, identify important dimensions, when to truncate decomposition. Low order smolyak grids needed is has been shown. First order should be good most of time...especially in nuclear applications where adjoints have been so successful.

\subsubsection{Combinatorics Routines} \label{subsubsec:combinatorics_routines}

In order to implement \ref{eq:decomp_smol3} on a computer several enumeration routines need to be available. Unfortunately, these routines are not available in most numerical math libraries containing combinatorics routines. The first routine of interest solves the problem of how to enumerate all the ways $d$ positive integers can be summed to equal another integer. In other words, what are all the sets $\left\{i_1,...,i_d\right\}$ such that $i_1 + i_2 + ... + i_d = q$? This problem inserts itself in \ref{eq:decomp_smol3} in the summation index for Smolyak interpolation. As the Smolyak interpolant becomes refined from level to level---$q$ is increased by one in each refinement---the Smolyak algorithm must newly account for all $\textbf{i}$ such that $\vert\textbf{i}\vert=q$. The following enumeration algorithm, a slight modification of the original algorithm found in \cite{Holtz}, finds all the desired indice sets:        
\begin{program} 
\begin{code}
# Initialize
p = 0
m = q - d + 1 
k = [0,1,...,1]   # vector of length d
kh = [m,m,...,m]  # vector of length d

# Begin algorithm...
repeat until k = [0,...,0]:
   k(p) = k(p) + 1
   if k(p) > kh(p):
      if p == d:
         All indices enumerated!
      else:
         k(p) = 1
         p = p + 1
   else:
      for j in range(p):
         kh(j) = kh(p) - k(p) + 1
      k(0) = kh(0)
      p = 1
      Return valid index set k!
\end{code}
\caption{\label{code:enumeration1} 
For postitive integers $d$ and $q$ this code outputs all sets $\left\{i_1,i_2,...,i_d\right\}$ such that $i_1+i_2+...+i_d=q$.}  
\end{program}

With all the index sets for some Smolyak level $q$ available through the code segment in \ref{code:enumeration1}, the tensor product appearing in \ref{eq:Smolyak_formula2} can be evaluated with the aid of an additional enumeration routine. Each indice in an index set $\left\{i_1+i_2+...+i_d\right\}$ corresponds to certain number of knots, which for example, is given by \ref{eq:cc_numpoints} for Clenshaw-Curtis. All the components in the tensor product can be given by the following enumeration algorithm, which is based on the algorithm in \cite{Holtz}. Input to the algorithm in code \ref{code:enumeration2} should be based on output from code \ref{code:enumeration1}. Specifically for some index set $\left\{i_1,i_2,...,i_d\right\}$ returned by code \ref{code:enumeration1}, each indice should be converted to a corresponding number of knots and input to code \ref{code:enumeration2}.     
\begin{program} 
\begin{code}
   # Initialize
   p = 0
   s = [0,1,1,...,1]  # vector of length d 

   # Begin algorithm...
   repeat until s == m: 
      s(p) = s(p) + 1
      if s(p) > m(p):
         if p == d-1:
            All indices enumerated!
         else:
            s(p) = 1
            p = p + 1
      else:
         p = 0
         Return valid enumeration set!
\end{code}
\caption{\label{code:enumeration2} 
Code for enumerating all components of a tensor product. The input is a $d$ dimensional vector $m$ where each entry $m_j$ corresponds to the number of knots in a collocation scheme of level $i_j$.}  
\end{program}

\subsubsection{Sampling Sparse Grid Interpolant of Correlated Variables} \label{subsubsec:sampling_smolyak}

ROM is just a collected of sparse grids. Built on assumption they are independent. How to sample with covariance matrix when they are not independent. Easy to evaluate so sampling should be very quick. Make sure in bounds. 


