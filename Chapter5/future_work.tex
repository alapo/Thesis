The research conducted in this thesis can be extended in many ways, in both the application and theory of surrogate models. In addition, the results of the research are suggestive of areas in which fission gas release modeling can be improved. The most direct extension of this research is to apply it towards any of the fuel performance models in the FUMEX databases \cite{fumex2} \cite{fumex3} that contain experimental data such as the Halden IFA  rods, Ris\o~GE7, OSIRIS and REGATE test cases. Surrogates can be constructed for any of these models using the framework discussed in this thesis. The surrogates can then be folded together with available experimental data to analyze thermal responses during power ramps, fission gas release, fuel pellet swelling and pellet clad mechanical interaction. Surrogates for these models would allow for sensitivity analyses, which would allow modelers to gain valuable insights into underlying physics. 

As scientists and engineers begin to write computer codes that are able to better emulate the fundamental physics in the material, thermalhydraulic, and neutronic components of nuclear processes surrogates should eventually be constructed for resulting multiphysics codes. Before such a task is undertaken it is essential to validate the individual codes to see if they are indeed capable of reproducing desired physics. Otherwise, coupling various codes  will have no effect on the efficacy of the multiphysics systems and any surrogates built upon it since its success will only be as strong as its weakest constituent. Such a coupled, multiphysics code system is currently being developed and validated in \ac{MPACT} \cite{Kochunas}. If a coupled code system is more accurate than any of its constituent pieces individually then a surrogate model for the multiphysics system should be constructed. Performing optimization and calibration with such a surrogate and experimental data should yield more accurate parameter analyses. However, it should be noted that a more accurate objective code will not alleviate a resulting surrogate from the problems that have classically plagued surrogates. Namely, the existence of many local optima. The extension of surrogates to multiphysics computer codes is a worthwhile investigation.  

In many of the problems where engineering computer codes are trying to model some physical phenomena there exists sparse experimental data to validate the codes. For example, in this thesis only a single time series measurement existed to measure fission gas release during the Ris\o~AN3 power ramp. With the availability of more experimental data, more interesting and rigorous analyses can be conducted with surrogate models. Error bars on experimental data enable probabilistic inferences of parameter values in calibration. As in \cite{Wang}, Bayesian hierarchical models can be used to probabilistically infer parameter values and even reconstruct temperature and flux fields. Such Bayesian approaches have the advantage of indicating the likelihood of parameters taking on certain values using posterior probability distributions, which can then be used for probabilistic risk assessment. Given the relatively limited amount of experimental data available in the nuclear engineering field, application of hierarchical Bayes for solving stochastic inverse problems should strongly be considered because it allows for the natural incorporation of multiple data sources. In other words, multiple experimental results, such as both pellet elongation and fission gas release data for the same problem, can be folded together to provide a holistic calibration approach.     

Another suggested area of research that would have wide ranging consequences for all engineering fields utilizing surrogate models would be in the development of a systematic method for initially reducing the size of the input parameter space. Despite claims to the contrary, existing surrogate construction methods can only handle $\mathcal{O}(10)$ variables. Unfortunately, most computer codes, and especially multiphysics systems, have hundreds if not thousands of inputs. A method is needed to reduce the initial parameter space to one more suitable for surrogate construction. For the problem in this thesis, the \ac{SIFGRS} model itself consisted of some dozen input parameters. Modeling the effects of all input parameters into Bison would have been a challenging task. Fortunately, much research had been conducted previously as to which fission gas release parameters generally have a large impact. Consequently, the initially large parameter space faced in this research was narrowed down by a thorough literature review. Ideally a mathematically rigorous method should exist to replace the literature review that is not immensely computationally burdensome.  

Finally, as evidenced by the calibrated fission gas kinetics in Fig. \ref{fig:fgr_best_estimate}, the Bison code fundamentally predicts fission gas release in a way different than experiment shows. As discussed in \cite{Pastore}, the discrepancies are likely due to the trapping and sudden release of fission gases in fuel grain cracks \cite{Pastore}. Efforts should be made to model this phenomena in \ac{SIFGRS} and the Bison code. Another area of development that may lessen any differences between predicted fuel performance metrics and observed values is the modeling of measurement devices. In this research, both thermocouples and pressure transducers are used to gather the experimental data. These instruments affected the quality of the experimental data and even the shape of the fission gas kinetics. Efforts should be made in Bison to incorporate the effects of instrumentation on predicted values in order for valid comparisons to be made between predicted and experimental data.        

