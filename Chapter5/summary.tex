Advances in computing the past decade have motivated scientists and engineers to write computer code that models physical phenomenon using as few approximations as possible. While the predictive capabilities of such codes' simulations improve in representing experimental data, they typically require hours on parallel compute clusters to complete. Indeed, from a statistical vantage point the analysis of computer experiments landscape has not changed. Engineering computer codes are still expensive to execute and therefore, techniques must be used to somehow simplify the codes before analyzing them. Newly available computational power has been spent towards making individual computer codes more accurate and encompassing rather than on easing the ability to perform optimization, calibration and sensitivity analysis in hopes of gleaning insight into the physics at hand. The purpose of this thesis has been to apply and develop so-called surrogate methods towards the statistical analysis of expensive engineering computer codes when the inputs to the codes are uncertain. 

Two surrogate approaches are investigated in this thesis. The first, anchored-\ac{ANOVA} decomposition on Smolyak sparse grids is a relatively new approach researched in the 2000s. Kriging is the second approach taken to analyze the output of computer experiments and has been considered the bread-and-butter of the field since the 1950s. In Kriging, a computer code is treated as a stochastic process. For various inputs, the computer code's outputs are observed and a statistical model is built based on the distance between input sets and the marginal differences in output. Usually \ac{LHS} is used to sample computer codes in an optimal fashion when only a limited number of computer simulations can be afforded. The anchored-\ac{ANOVA} decompositions takes a more deterministic approach towards surrogate building. First, a computer code is sliced into components of first order, second order, and higher order combinations of input variables. For each component a polynomial interpolant is built on a Smolyak sparse grid. In the anchored-\ac{ANOVA} approach one does not have the flexibility to evaluate the computer code at any arbitrary points. Rather, the computer code must be evaluated at points dictated by the collocation set and convergence criteria chosen. Based on the insight that higher order interaction effects are small compared to their first order counterparts, very accurate surrogates can be constructed with relatively few points. 

Initially, both surrogate approaches were applied to text-book type problems in order to gain familiarity and to identify which approach would be more suitable when applied to modeling fission gas release using a computationally expensive computer code. In the anchored-\ac{ANOVA} approach each variable considered requires special attention and consequently, for simple problems with relatively small uncertainties Kriging was observed to build as accurate of surrogate models as anchored-\ac{ANOVA} using less objective function simulations. Spectral convergence was observed for each text-book problem considered when the anchored-\ac{ANOVA} approach was applied. However, the number of new simulations of the objective function for each increasing level of Smolyak interpolants grows exponentially. In the classic formulations of both surrogate methods the input parameters are assumed to be independent of one another. For statistical analysis this assumption can be circumvented by building the surrogates with the independent assumption and then sampling the surrogates using a covariance matrix that describes the variables' interconnectedness. The performance of Clenshaw-Curtis and Gauss-Patterson collocation sets were both investigated in application to Smolyak interpolants. For the problems in this thesis, the higher theoretical accuracy of Gauss-Patterson did not show and consequently, Clenshaw-Curtis collocation is recommended due to its transparency and near-optimal performance. 

Ultimately, Kriging was chosen as the surrogate approach to apply towards modeling fission gas release in Bison. The decision resulted from consideration on several fronts. The non-linearity of fission gas release models coupled with large uncertainties implied the need for modeling higher-order interaction effects with the surrogate. As indicated by the text-book problems, modeling such higher-order effects with anchored-\ac{ANOVA} and Smolyak sparse grids can get very expensive, with no clear limit of how many objective function simulations will be needed to complete the surrogate. A Kriging surrogate makes more sense when faced with a limited computational budget. In addition, the transparency of Kriging was appealing when considering each Bison fission gas release simulation would have to be performed in parallel. For Kriging a total of $n$ randomly sampled simulations are needed in order to construct the surrogate and thus, if one simulations fails to converge or experiences an error, the correction process is straight forward. Contrarily, there are a lot of moving pieces in the anchored-\ac{ANOVA} surrogate approach. Repairing the damages to a surrogate's construction due to a failed simulation requires much book keeping. Finally, Kriging was chosen over the anchored-\ac{ANOVA} approach for fission gas release due to Kriging's clear extension to time series. Both surrogate methods investigated in this thesis were primarily developed for scalar quantities. It was not clear how the anchored-\ac{ANOVA} approach could be extended to construct a surrogate for a time series without taking on an immense computational expense.     

Before investigating fission gas release kinetics for the Ris\o~AN3 power ramp using Kriging, eight \ac{SIFGRS} parameters were chosen for analysis. The eight parameters were the initial fuel grain radius, fuel porosity, bubble surface tension, temperature, fuel grain radius, intra-granular gas atom diffusion coefficient, vacancy diffusion coefficient and resolution parameter. Uniform distributions spanning orders of magnitudes were assigned to the parameters in order to reflect their large uncertainties. A total of 100 \ac{LHS} of the parameters were then taken and propagated through Bison and the Ris\o~AN3 fission gas release model. Principal component analysis was applied to model the variability in the fission gas kinetics output. Only three principal components were necessary to capture over 99\% of the variance in the 100 fission gas release time series.    

Kriging was extended to produce a surrogate for entire time series by constructing individual scalar Kriging surrogate models for the expansion coefficients of each of the three principal components. The time series Kriging model was then cross validated before applying it to calibrate Bison's \ac{SIFGRS} parameters to experimental data. Although a locally optimal solutions was found by minimizing \ac{RMSE}, there were apparent and irreconcilable differences between Bison's fission gas release predictions and the experimental data. While experimental data lays mostly outside of the calibrated fission gas kinetics output's 99\% confidence intervals, the end-of-experiment error was under 3\%. Some of the differences between prediction and experiment can be attributed to several fission gas release aspects not explicitly modeled in Bison. Namely, burst fission gas release due to micro-cracking and the effect of measuring fission gas release using pressure transducers. Experiments have identified scenarios where released fission gases get trapped in closed gaps and cracks in the fuel, thereby not contributing to changes in plenum pressure that can be measured by pressure transducers. Another factor contributing to discrepancies between predicted and experimental fission gas release time series is the uncertainty in \ac{SIFGRS} parameters not modeled in this thesis.    

Sobol indices were calculated for the fission gas release parameters using the Kriging surrogate. The starting fuel grain radius and fuel temperature had the highest sensitivity indices and produced the largest non-linear interaction effects with the other parameters. While sensitivity coefficients were calculated for the Ris\o~AN3 problem, it is not certain that the same parameter conclusions would generalize to other fission gas kinetics problems. The same type of analysis as conducted in this thesis would likely have to be replicated for each unique problem. The lack of generalization is likely due to the unique profundity of physics in play for each type of problem. 

The original contributions of this thesis are three fold. First, the construction of a surrogate model for the fuel performance code Bison and subsequent calibration of fission gas release parameters to experimental data from the FUMEX database. Second, the extension of Kriging to construct surrogates for entire time series through \ac{PCA}. Finally, the application of anchored-\ac{ANOVA} decomposition and Smolyak sparse grids to construct surrogates for classic nuclear engineering problems.


  